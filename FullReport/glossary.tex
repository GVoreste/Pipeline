\newglossaryentry{RISCV}
{
    name=RISCV,
    description={\href{https://riscv.org/about/}{RISC-V} is an open standard \ISA}
}

\newglossaryentry{bubble_sort}
{
    name=bubble sort,
    description={Sometimes referred to as sinking sort, is a simple sorting algorithm that repeatedly steps through the input list element by element, comparing the current element with the one after it, swapping their values if needed \cite{wiki:bubble}}
}

\newglossaryentry{clk}
{
    name=clock cycle,
    description={Also known as a machine cycle or a clock tick, is the basic unit of time in a computer's \acrshort{cpu} and synchronous electronic}
}

\newglossaryentry{stall}
{
    name=stall,
    description={In the design of pipelined computer processors, a pipeline stall is a delay in execution of an instruction in order to resolve a \gls{hazard} \cite{wiki:stall}}
}

\newglossaryentry{hazard}
{
    name=hazard,
    description={In the domain of \acrfull{cpu} design, hazards are problems with the instruction pipeline in \acrshort{cpu} microarchitectures when the next instruction cannot execute in the following \gls{clk}, and can potentially lead to incorrect computation results. Three common types of hazards are data hazards, structural hazards, and control hazards (branching hazards) \cite{wiki:haz}}
}

\newglossaryentry{linking}
{
    name=linking,
    description={The act of 'linking' intended as the action on the code performed by the \gls{linker}}
}
\newglossaryentry{linker}
{
    name=linker,
    description={In computing, a linker or link editor is a computer system program that takes one or more object files (generated by a compiler or an assembler) and combines them into a single executable file, library file, or another "object" file
    \cite{wiki:link}}
}
\newglossaryentry{control_sig}
{
    name=control signal,
    description={In a \acrshort{cpu} is a signal that select the path of the data in the different stages of execution of the instruction. They determine for example the operations performed by the different components and the selection of the \acrshort{mux}s. They can be viewed as the signals that hold and transport the internal data of the \acrshort{cpu}
    }
}
\newglossaryentry{multiplexer}
{
    name=multiplexer,
    description={
    In electronics, a multiplexer (or mux; spelled sometimes as multiplexor), also known as a data selector, is a device that selects between several analog or digital input signals and forwards the selected input to a single output line. The selection is directed by a separate set of digital inputs known as select lines. A multiplexer of $ 2^{n}$ inputs has n $n$ select lines, which are used to select which input line to send to the output \cite{wiki:mux}
    }
}






\newacronym{ram}{RAM}{random-access memory}
\newacronym{cpu}{CPU}{central processing unit}
\newacronym{pc}{PC}{programm counter}
\newacronym{isa}{ISA}{istruction set architecture}
\newacronym{rom}{ROM}{read-only memory}
\newacronym{mux}{MUX}{\gls{multiplexer}}
\newacronym{alu}{ALU}{arithmetic logic unit}
